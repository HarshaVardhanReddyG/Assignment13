\documentclass{beamer}
\usetheme{CambridgeUS}

\setbeamertemplate{caption}[numbered]{}

\usepackage{enumitem}
\usepackage{tfrupee}
\usepackage{amsmath}
\usepackage{amssymb}
\usepackage{textcomp, gensymb}
\usepackage{graphicx}
\usepackage{txfonts}

\def\inputGnumericTable{}

\usepackage[latin1]{inputenc}                                 
\usepackage{color}                                            
\usepackage{array}                                            
\usepackage{longtable}                                        
\usepackage{calc}                                             
\usepackage{multirow}                                         
\usepackage{hhline}                                           
\usepackage{ifthen}
\usepackage{caption} 
\providecommand{\mbf}{\mathbf}
\providecommand{\qfunc}[1]{\ensuremath{Q\left(#1\right)}}
\providecommand{\sbrak}[1]{\ensuremath{{}\left[#1\right]}}
\providecommand{\lsbrak}[1]{\ensuremath{{}\left[#1\right.}}
\providecommand{\rsbrak}[1]{\ensuremath{{}\left.#1\right]}}
\providecommand{\brak}[1]{\ensuremath{\left(#1\right)}}
\providecommand{\lbrak}[1]{\ensuremath{\left(#1\right.}}
\providecommand{\rbrak}[1]{\ensuremath{\left.#1\right)}}
\providecommand{\cbrak}[1]{\ensuremath{\left\{#1\right\}}}
\providecommand{\lcbrak}[1]{\ensuremath{\left\{#1\right.}}
\providecommand{\rcbrak}[1]{\ensuremath{\left.#1\right\}}}                                  
                               
\title{Assignment 13}
\author[CS21BTECH11017]{G HARSHA VARDHAN REDDY ( CS21BTECH11017 )}
\date{\today}
\logo{\large{AI1110}}


\begin{document}
% Title page frame
\begin{frame}
    \titlepage 
\end{frame}

% Remove logo from the next slides
\logo{}


% Outline frame
\begin{frame}{Outline}
    \tableofcontents
\end{frame}

%Question
\section{Problem Statement}
\begin{frame}{Problem Statement}
    \begin{block} {Papoulis Pillai Probability Random Variables and Stochastic Processes\\ 
    Exercise : 8-10}
    Among $4000$ newborns, $2080$ are male. Find the $0.99$ confidence interval of the probability
$p = P\{male\}$.
    \end{block}
\end{frame}
\section{Definitions}
\begin{frame}{Definitions}
    \begin{block}{Sample Proportion}
     If $X$ is a binomial random variable, then $X \sim B(n, p)$ where n is the number of trials and $p$ is the probability of a success. To form a sample proportion, take $X$, the random variable for the number of successes and divide it by $n$, the number of trials (or the sample size). The random variable $P'$ is the sample proportion
     \begin{align}
         P' &= \frac{X}{n} 
     \end{align}
And \\
    $p'$ =  the estimated proportion of successes or point estimate for $p$
    \end{block}
\end{frame}

\begin{frame}{Youth percentile or Z score}
    \begin{block}{Z-score}
       Z-score indicates how much a given value differs from the
    standard deviation. The Z-score, or standard score, is the number of standard deviations a given data point lies above or below mean.
    \begin{align}
        \implies Z_u = \frac{x-\mu}{\sigma} = \frac{p-p'}{\sigma_{p'}}
    \end{align}
     Where,\\
        $Z_u$ = Normal (Youth) percentile or $Z$ score\\
        $ x $ = Observed value\\
        $\sigma $ = Standard deviation
        
    \end{block}
\end{frame}
\begin{frame}{}
     \begin{block}{Confidence interval for a population proportion}
        The confidence interval for a population proportion $(p)$
        \begin{align}
        | p-p' | \leq \sigma Z_u\\
        p'-\sigma Z_u \leq p \leq p'+\sigma Z_u
        \end{align}
        Where
        \begin{align}
            \sigma_{p'} &= \sqrt{\frac{(1-p')(p')}{n}}
        \end{align}
    Therefore,
        \begin{align}
           p' - Z_u \times \sqrt{\frac{(1-p')(p')}{n}} \leq p \leq p' + Z_u \times \sqrt{\frac{(1-p')(p')}{n}}\label{6}
        \end{align}
    \end{block}
\end{frame}
\section{Solution}
\begin{frame}{Solution}
    Given,\\
    \begin{align}
        \text{No.of newborns } (n) &= 4000 \label{eq7}\\
        \text{No.of males } &= 2080\\
        \implies p'=P\{male\} &=\frac{2080}{4000}=0.52\label{eq9}\\
        \text{Confidence coefficient }(CF) &= 0.99\\
        \implies Z_u &=2.326 \label{eq11}
    \end{align}
$\therefore$ From \eqref{6},\eqref{eq7},\eqref{eq9} and \eqref{eq11},
\begin{align}
    0.52 -2.326\sqrt{\frac{(0.48)(0.52)}{4000}} \leq\text{ } &p \text{ } \leq 0.52 +2.326\sqrt{\frac{(0.48)(0.52)}{4000}}\\
    \implies 0.502 \leq \text{ } &p \text{ } \leq 0.538
\end{align}
    
\end{frame}
\end{document}